%\section{Implementation of the new partial-sum attack on 6 rounds \AES}\label{sec:attack_aes}
%\begin{algorithm}[t]
    \begin{algorithmic}[1] % enter the algorithmic environment
        \State Input: Array $\mathcal{X}$ of bits such that the $j-th$ value of $\mathcal{X}$ denotes the parity of ciphertexts $j$
        
        \State Declare an empty 2D bit-array $A[][]$ of size $2^{24}\times2^{16}$; \Comment{$2^{40}$ memory}
         
            \For{$c_3c_2a=0;c_3c_2a<2^{24};c_3c_2a=c_3c_2a+1$}
            \tcb{
                \For{ $k_1k_0=0;k_1k_0<2^{16};k_1k_0=k_1k_0+1$ }
                    \State 
                         {\tiny  $ \displaystyle A[c_3c_2a][k_1k_0] = 
                        \sum_{ c_1c_0 \in \{0,2^{16}-1\} } \mathcal{X}_i[c_3c_2c_1c_0]\cdot \mathcal{I}(a = S_0(c_0 \oplus k_0) \oplus S_1(c_1 \oplus k_1) )$}
                \EndFor
                    }
            \EndFor
        \For{$k_1k_0 = 0; k_1k_0 < 2^{16}; k_1k_0 = k_1k_0 + 1$}
            \State Declare an empty 2D bit-array $B[][]$ of size $2^{8}\times2^{16}$;
                \For{$c_3=0;c_3<2^{8};c_3=c_3+1$}
                \tcb{
                    \For{ $k_2b=0;k_2b<2^{16};k_2b=k_2b+1$ }
                    \State 
                    {\tiny $B^{\prime}[i][c_3][k_2b] = 
                        \sum_{ c_2a \in \{0,2^{16}-1\} } \mathcal{A}[i][k_1k_0][c_3c_2a]\cdot \mathcal{I}(a \oplus b = S_2(c_2 \oplus k_2))$}
                    \EndFor
                }
                \EndFor
            \For{$k_2 = 0; k_2 < 2^{8}; k_2 = k_2 + 1$}
                \State Declare an empty 1D bit-array $\mathcal{C}[]$ of size $2^{16}$;
                    \tcb{
                    \For{ $k_3c=0;k_3c<2^{16};k_3c=k_3c+1$ }
                    \State 
                    { $\mathcal{C}[i][k_3c] = 
                        \sum_{ c_3b \in \{0,2^{16}-1\} } \mathcal{B}[i][k_2][c_3b]\cdot \mathcal{I}(c \oplus b = S_3(c_3 \oplus k_3))$}
                    \EndFor
                    }
                    \For{$k_3 = 0; k_3 < 2^{8}; k_3 = k_3 + 1$}
                    \State Declare an empty 1D bit-array $\mathcal{S}[]$ of size $2^{8}$;
                        \For{$i=0;i<d;i=i+1$} 
                        \tcr{
                            \For{ $k_4=0;k_4<2^{8};k_4=k_4+1$ }
                            \State 
                            $\mathcal{S}[k_4] = 
                                \sum_{ c \in \{0,2^{8}-1\} } \mathcal{C}[k_3][c]\cdot S^{-1}(c\oplus k_4))$
                            \EndFor
                        }
                       \EndFor
                       \For{$k_4 = 0; k_4 < 2^{8}; k_4 = k_4 + 1$}
                            \If{$\mathcal{S}[k_4] \neq 0$}
                                \State $RK[k_4k_3k_2k_1k_0] = 1$
                            \EndIf
                       \EndFor
                \EndFor
            \EndFor
        \EndFor
    \end{algorithmic}
\caption{The following is the Algorithm for key recovery. The function $\mathcal{I}$ is the indicator function. All the \tcb{blue} colored steps are of complexity $2^{16}\times2^{16}$ and can be replaced by a $3$ Hadamard transformations of size $2^{16}$ with total complexity $3 \times 2^{20}$. The \tcr{red} colored step has complexity $2^8 \times 2^8$, which can be replaced by $3$ Hadamard transformations of size $2^{8}$ with total complexity $3 \times 2^{11}$. Note that we need XOR of $8$ bit values from the last Hadamard transformation. Here I have used Gaetan's trick with \tcr{$7$} bit separation that needs exactly $64$ bits. \label{alg:fht}}
\end{algorithm} 

%\subsection{Explantion of Gaetan's trick}
%Here let us explain the last \tcr{red} color $\mathsf{for}$ loop in the Algorithm~\ref{alg:fht}. We are basically computing a convolution of two function $f:\mathbb{F}_2^8 \to \mathbb{F}_2$ and $g:\mathbb{F}_2^8 \to \mathbb{F}_2^8$ such that
%\begin{align*}
%    f(c) = \begin{cases}
%       1, \text{ if $\mathcal{C}[i][k_3][c] = 1$ for some fixed $i$ and $k_3$}\\
%       0, \text{ otherwise} 
%    \end{cases}
%\end{align*} 
%and
%\begin{align*}
%    g(x) = S^{-1}(x).
%\end{align*} 

%So the convolution we are interested in is $f*g(k) = \sum_{c \in \mathbb{F}_2^8} f(c)g(c \oplus k)$ (However, in the end we need sum over $\mathbb{F}_2^8$, not over $\mathbb{Z}$, will deal with it later). We need to compute the above convolution for all $k \in \mathbb{F}_2^8$.

%Computing the convolution for all $k$ is actually a matrix-vector multiplication. The vector is $V = (f(0),....,f(2^8-1))$ and the matrix is $M$ where $M(i,j) = g(i\oplus j)$. Beacuse of the \emph{blockwise circular} property of the matrix we have the following equation
%\[
%M\cdot V = (1/2^8)H(H(C_1)\times H(V)) 
%\]
%where $H$ is a $2^8 \times 2^8$ Hadamard matrix and $C_1$ is the first column of $M$. These are the three Hadamard transformations that we need to compute.

%As the Hadamard matrix is over $\mathbb{Z}$ and not over $\mathbb{F}_2^8$, we can not compute the XOR sum as desired. So Todo's idea is to consider $g(x) = (g_0(x),...,g_7(x))$ and compute eight matrix-vector multiplications separately and finally combine them. Here Gaetan suggested considering a single matrix; however, each bit of each matrix element is separated by a few bits. For example if $M_{0x5,0x6} = g(0x5\oplus 0x6) = 0xFF$ and we consider $7$ bits seperation, then we write this as
%{\small$$00000000-0000001-0000001-0000001-0000001-0000001-0000001-0000001-0000001$$}

%\subsection{Experiment Detail}
%I tested the idea with random vectors $V$. I compute the following two things:
%\begin{itemize}
%   \item $U_1 = MV$, i.e., the convolution directly
%   \item $U_2 = (1/2^8)H(H(C_1)\times H(V))$ with $6,7$ or $8$ bit seperation
%\end{itemize}
%With a $7$ bit separation, I am always getting $U_1 = U_2$. However, for $6$ and $8$ bit separation I am always getting some positions of $U_1$ and $U_3$ are not equal.
%\subsection{Description of the attack}\label{sub:description_new_attack}

%The new attack with 6 $\Delta$-sets

%\subsection{Complexity analysis}\label{sub:complexity_new_attack}


%\subsection{Reducing the number of $\Delta$-sets}\label{sub:complexity_new_attack}



%\subsection{Experimental verification}\label{sub:experimental_verification}

